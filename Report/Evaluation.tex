\section{Evaluation}

One of the easiest ways to evaluate the success of the project is prove that we are able to trace function calls as we discussed in the Introduction. Given a target program, we want to be able to print out all occurrences of invocations of its functions. A specific example we want to test against is another lighttpd bug\footnote{Bug \#2169 (http://redmine.lighttpd.net/issues/2169)}. Although this bug is currently resolved, the original issue took around 8 months to locate and fix. Our library should allow us to instrument an old version of lighttpd to provide logging of:

\begin{enumerate}
 \item \textbf{Procedure invocations} - We should be able to gather information about the function calls leading up to the bug occurring. This will require \emph{(F1)} and \emph{(F2)}.
 \item \textbf{Procedure return values} - If we are able to find the call convention used for the application, we will be able to instrument code at the return address of functions to read off a particular register's value which should be holding the return value. This will require \emph{(F3)}.
 \item \textbf{Procedure arguments} - If we instrument code at the start of a function, we should be able to inspect the stack and log the values of the arguments.
\end{enumerate}

\emph{(F4)}, \emph{(F6)} and \emph{(F7)} are implicit requirements, but it is possible for us to evaluate \emph{(F5)} by targeting the library against two versions of lighttpd (before and after fix), and accessing the generated CFG to print out the code differences.

Since our approach is similar to EEL's, we can also measure our success by quantitatively comparing the overhead added by the library. As we saw earlier, EEL creates a markup from 8KB to 350KB. We believe we can make an improvement on this and we can easily compare results as a way of evaluating our library.