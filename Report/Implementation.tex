\chapter{Implementation}



\chapter{Project Plan}

In order to complete the project, we need to satisfy all of the functional requirements. Given that we already have some form of plan for the implementation of the library, we can come up with some milestones and an approximate timeline:

\begin{table}[H]
 \caption{Projected timetable for project}
 \begin{tabular}{l p{12cm} }
 \hline
 Expected By & Milestone \\
 \hline
 Mid-Feb & Be able to perform static analysis of a program to generate the CFG \\
 End-Feb & Be able to inject an object file into targets \\
 End-Mar & Be able to perform regular detouring \\
 Mid-Apr & Be able to perform trampolining \\
 End-Apr & Perform evaluation and write user guide\\
 End-May & Write up the rest of the report \\
 \hline
 \end{tabular}
\end{table}

\chapter{Extensions}

If time permits, the most desirable extension is to add dynamic detouring to the library. We would implement some form of runtime detouring and provide users with the option to choose between both methods. Another smaller extension which would also be useful is to allow the library to be invocable through the command-line as well as through the API (as seen with ELFsh).

% disasm engine builds on top of several abstractions. detailed diagram of disasm engine and how it works above libbfd and libopcodes. perhaps some justification for why we chose to use these
%
%in practice, the disassembler engine is responsible for parsing and translating the strings received from libopcodes and storing that information in its internal semantic representation (bf\_insn, bf\_basic\_blk). bf\_func can be identified from three ways only:
%1) if it is the target of a call site
%2) if it corresponds to an address identified as a bf\_sym
%3) if the user defined it as a root for cfg generation and explicitly stated it is a function (cfg analysis and generation is covered in depth in...)
%
%further information such as size of symbols (which tells us size of function) is available because we are parsing the symbol table directly from libelf.
%
%quirks of libopcodes disassembler such as how it passes us the instruction parts (mnemonic, operands, separator) as strings! we build a finite state machine which allows us to 
%
%we can draw finite state machine in diagram here
%
%binary\_file
%
%within a binary\_file, there are 2 levels of code representation. firstly we have bf\_basic\_blk which represents a basic\_block as defined in architecture.
%
%binary\_file is composed of a CFG of bf\_basic\_blk objects. during the process of cfg generation and after it completes, we add extra information. i.e., bf\_func 'labels'
%
%bf\_func
%
%implementation of CFG analysis and generation...
%
%epilogue relocation
%
%distribution and documentation
%
%automake which we use to deal with library dependencies (libelf, libbfd, libopcodes, libkern).
%
% and doxygen
%
%WORKFLOW